\documentclass[a4paper,10pt]{article}
\usepackage[utf8x]{inputenc} 

%opening
\title{Group 12 Project Proposal and Requirements Document}
\author{Eric Momper, Peter Lomason, John Barber}
\setlength{\parindent}{0pt}
\begin{document}

\maketitle

\pagebreak
\tableofcontents
\pagebreak

\section{Introduction}
Our Project is intended to be a psychological theraputic tool that helps patients through simulation in Virtual Reality.
Many of the new up and coming Virtual Reality devices look very promising at providing better and more realistic immersion and simulation for users. 
\\
\paragraph{A few of these devices include:}
\begin{itemize}
 \item Occulus Rift, Occulus Touch
 \item HTC Vive
 \item Samsung Gear VR
 \item Google Cardboard
 \item Microsoft Hololens
\end{itemize}

\paragraph{ Some of our project ideas are related to the following:}
\begin{itemize}
 \item ​Immersion Therapy for phobias, dysphorias, or PTSD (Virtual or Augmented Reality)
 \item Therapy for burn victims, phantom pain for amputees,  (Augmented Reality)
 \item Creating a calm environment for  anxiety disorders, or autism (Virtual Reality)
 \item Creating a drawing art therapy tool that allows for creativity in 3D space. 
\end{itemize}


\paragraph{Project Direction} ~\\ Are currently trying to contact faculty in the psychology department for collaboration or 
guidance on opportunities to use our project for helping their patients or for their research.
\paragraph{Integration} ~\\ Project will be created in Unity, Unnreal Engine, an Android App, or written with a C++ OpenGL/ DirectX SDK wrapper, depending on the direction of our design, and 
the device we can use. It will most likely be windows exclusive as many VR devices are dropping Linux and OSX support, unless we can find some options that will 
support cross platform systems, crossplatform support is said to be available in the near future but is not currently finished.  

\pagebreak

\section{Past Research}
studies go here...
\subsection{Immersion Therapy}
AR
\subsection{Pain Treatment}
AR, VR
\subsection{Creating a Calming Environment}
AR
\subsection{Art Therapy}
AR, VR
\pagebreak

\section{Team Member Motivations}
\subsection{Eric Momper}
My personal motivation for this project is that it is similar to some of the programming I do on my internship (OpenGL and OpenCL image GPGPU processing).
I also have taken computer graphics with Professor Leinicker and I am currently taking robot vision with Doctor Lobo. This project will be very interesing to me as  
I will be working with new technologies and programming on a new type of 3D graphics platform. I am also looking foward to studying the psychological benefits
of Virtual Reality on patients with various disorders.  

\subsection{John Barber}
stuff here
\subsection{Peter Lomason}
My interest in this project mainly comes from past experience with virtual reality tools. I used to own an Oculus Rift DK1 and at the time I had it, I was not knowledgeable enough to develop for it. 
\section{Goals and Objectives}
\section{Function}
Design criteria and Constraints
\section{Broader Impacts}
Broad implications and impact on society (impact on underrepresented? within STEM and or society as a whole? disabled? non-profit orgs? environment? diversity? increased participation in STEM or workforce? public engagement in STEM? improve national security? enhanced infrastructure? improved education?)

qualitative, avoid numbers, conceptual discussion specific to project.
example descriptions "“lightweight, portable, programmable, low cost, flexible, high resolution, scalable, low power, accurate, mobile, peer-to-peer, autonomic”


\section{Requirements}
numbers bitch, quantitative measures and metrics, verify and evaluate
how many,

how often, how high, how long, how complex, what values, what events occur, etc.
\subsection{Specs}
design reqs go here
\subsection{Charts}
project block diagrams - gantt chart goes here
prototype illustration?

\section{Budget and Financing}
software licensing costs, cloud based service costs, code repo's, graphic design costs
\section{Milestones}
\end{document}
